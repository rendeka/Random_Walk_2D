%%%%%%%%%%%%%%%%%%%%%%%%%%%%%%%%%%%%%%%%%
% Wenneker Assignment
% LaTeX Template
% Version 2.0 (12/1/2019)
%
% This template originates from:
% http://www.LaTeXTemplates.com
%
% Authors:
% Vel (vel@LaTeXTemplates.com)
% Frits Wenneker
%
% License:
% CC BY-NC-SA 3.0 (http://creativecommons.org/licenses/by-nc-sa/3.0/)
% 
%%%%%%%%%%%%%%%%%%%%%%%%%%%%%%%%%%%%%%%%%

%----------------------------------------------------------------------------------------
%	PACKAGES AND OTHER DOCUMENT CONFIGURATIONS
%----------------------------------------------------------------------------------------

\documentclass[11pt]{scrartcl} % Font size
\input{structure.tex} % Include the file specifying the document structure and custom commands

%----------------------------------------------------------------------------------------
%	TITLE SECTION
%----------------------------------------------------------------------------------------

\title{	
	\normalfont\normalsize
	\textsc{Faculty of Mathematics and Physics}\\ % Your university, school and/or department name(s)
	\vspace{25pt} % Whitespace
	\rule{\linewidth}{0.5pt}\\ % Thin top horizontal rule
	\vspace{20pt} % Whitespace
	{\huge Random walk in 2D square grid}\\ % The assignment title
	\vspace{12pt} % Whitespace
	\rule{\linewidth}{2pt}\\ % Thick bottom horizontal rule
	\vspace{12pt} % Whitespace
}

\author{\LARGE Andrej Rendek} % Your name

\date{\normalsize\today} % Today's date (\today) or a custom date

\begin{document}

\maketitle % Print the title

\section{Simple walk}

This is a walk, where all direction are equally likely at each point of the walk. We have simulated $N = 1600$ for a length of the walk $n = 2^x$, where $x \in \{2, \ 3, \dots, \ 16, \ 17\}$. For $N = 1600$ we have standard deviation $\sigma \sim \dfrac{1}{\sqrt{N}} = 0.025$ for each simulated value. We have fitted results of the simulation in the form:

\begin{equation*}
	\ln \left[R(n) \right] = \ln(c) + \alpha \ln(n),
\end{equation*}
where $R(n)$ is an Euclidean distance between the origin and the final position of the walk, $n$ is a length of the walk, and $c$ and $\alpha$ are fitting parameters. The dependence came out linear as expected, see \ref{fig:simple-fit}, fitted parameters are:

\begin{align}
	\alpha &= 0.507 \pm 0.004, \\
	c &= 0.884 \pm 0.025
\end{align}

\begin{figure}[h] 
	\centering
	\includegraphics[width=0.8\textwidth]{fig_walk_length.png} 
	\caption{Euclidean distance for simple walk.}
	\label{fig:simple-fit}
\end{figure}

\section{Non-returning walk}

For this walk, we have repeated the same procedure as in simple walk. 

\begin{align}
	\alpha &= 0.530 \pm 0.004, \\
	c &= 1.012 \pm 0.029
\end{align}

\begin{figure}[h] 
	\centering
	\includegraphics[width=0.8\textwidth]{fig_walk2_length.png} 
	\caption{Euclidean distance for non-returning walk.}
\end{figure}

\section{Non-crossing walk}

We simulated $N = 40000$ random walks with the length set to $n = 10000$, but all walks were terminated at most after $600$ steps. Here is an example of one of the longest walks \ref{fig:non-crossing}. The average length of this walk altered between $70-71$ as expected.

\begin{figure}[h] 
	\centering
	\includegraphics[width=0.8\textwidth]{fig_walk_long.png} 
	\caption{Example of non-crossing random walk.}
	\label{fig:non-crossing}
\end{figure}


\end{document}
